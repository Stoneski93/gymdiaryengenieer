\documentclass[twoside,12pt]{wipb}
\usetikzlibrary{mindmap,trees}%dla diagramu Computer science mindmap

\usepackage{url}
\usepackage{float}
\usepackage{paralist}
\usepackage{xcolor}
\usepackage[nottoc,numbib]{tocbibind}

\widowpenalty=10000
\clubpenalty=10000

\raggedbottom

\katedra{OPROGRAMOWANIA}
\typpracy{ inżynierska
           %inżynierska
         }

\temat{APLIKACJA MOBILNA DO ZARZĄDZANIA TRENINGIEM NA SIŁOWNI}
\autor{Norbert Kamieński}
\promotor{dr inż. Marcin Czajkowski}
\indeks{93401}

\studia{stacjonarne
        %niestacjonarne
       }

\rokakademicki{2016/2017}

\profil{%magisterskie jednolite
        %magisterskie uzupełniające
        %studia I stopnia
        studia I stopnia
}
\kierunekstudiow{informatyka
                 %matematyka
                }

\specjalnosc{ 
             %Inżynieria Komputerowa
             %Systemy Oprogramowania
             %Metody infnformatyczne w~banknkowości i~finansach
             %Ochrona systemów informatycznych
            }

\zakres{1. Opis technologii wykorzystywanych przy tworzeniu systemu.  \newline
 2. Wykonanie projektu aplikacji do zarządzania treningiem na siłowni.   \newline 
 3. Implementacja i testy zrealizowanej aplikacji.  \newline }
\slowakluczowe{Aplikacja Mobilna, React Native, Firebase, JavaScript, Android, Organizer }


\hypersetup{ %wpisy w pdf info
pdfauthor={Norbert Kamieński},
pdftitle={APLIKACJA MOBILNA DO ZARZĄDZANIA TRENINGIEM NA SIŁOWNI},
pdfsubject={Krótki opis pracy. },
pdfkeywords={Aplikacja Mobilna, React Native, Firebase, JavaScript, Android, Organizer},
pdfpagemode=UseNone,
linkcolor=black,
citecolor=black
} 

\begin{document}
\maketitle
\chapter*{Mobile application to manage training in the gym}

The main goal of this thesis was to create a mobile application that helps athletes manage their gym training sessions. The app allows for its users for an easy and convenient way to organize and track the progress of their workouts.

Human memory is an unreliable and athletes struggle with that as well. Trainees don’t remember the weight they have recently worked out with or how many series or repetitions they did.
 
Some write it down in a regular notebook. It is not very handy, though, might be forgotten at times and has limited space. Most people nowadays carry a mobile phone around everywhere, including gym. The application solves these and many others problems. It allows for a user to note progress during a training session that can be reviewed and used  at a later time.

Application has been developed using JavaScript framework React Native, supported by Redux technology. Framework uses functional approach to programming and frontends technologies that include HTML, JSX and CSS. Despite only web technologies being used, the application is 100\% native.

The paper contains five chapters. The first is an introduction. It presents the goal, plan and scope of work. Second chapter shows an overview of the application, existing solutions and functional and nonfunctional requirements. The most important is chapter four and it describes the development and implementation process. It contains application architecture, principle of operation, tests and product presentation.
 
The last chapter is the summary. \\
Finally there’s an app used by advanced athletes.



\tableofcontents
\thispagestyle{empty}
\setcounter{page}{0}
\pagestyle{plain}

\addtocontents{toc}{\protect\thispagestyle{empty}}

\chapter[Wstęp.]{Wstęp}

\section{Cel pracy}

\section{Zakres pracy}

\chapter[Analiza wymagań.]{Analiza wymagń}

\section{Wizja oraz ogólny zarys aplikacji.}
Głównym celem powstania aplikacji jest pomoc osobom trenującym w organizacji ich treningu na siłowni. 

Często spotykanym zjawiskiem jest brak pamięci do ostatnio wykonywanego treningu. Zwłaszcza początkujący ale i również doświadczone osobymają problemy z zapamiętaniem trenowanych ćwiczeń, przenoszonego obciążenia czy też zakresu powtórzeń.

Trenujący próbują rozwiązać ten problem zabieraniem na trening zeszytu oraz notowaniem progresu.
NIe rozwiązuje to jednak do końca problemu. NIerzadko przytrafia się zapomnienie zeszytu, długopisu a same targanie wraz z innymi akcesoriami treningowymi nie należy do przyjemnych. Nie posiada on również nieograniczonego miejsca oraz jest podatny na uszkodzenia.

Najodpowiedniejszym rozwiązaniem problemów wydaje się aplikacja mobilna. Jedyną rzeczą jaką użytkownik musi zrobić, to wziąć swój telefon na trening. Telefon służy nam w różnych aspektach życia co sprawia, że trudno go zapomnieć.

Aplikacja nie będzie skomplikowana oraz przeładowana. Posiadać będzie czytelny interfejs. Najważniejszą przewidzianą funkcją ma być archiwizacja danych oraz notowanie aktualnego treningu. Ważnym dodatkiem będzie lista ćwczeń zawierająca opis każdego z nich. Zakładka ulubione ćwiczenia- pomoże łatwiej je pogrupować oraz odszukać użytkownikowi. Rzeczą usprawniającą trening będzie minutnik, który pomoże precyzyjnie odmieżać czas wykonywanego interwału lub przerwy między seriami lub ćwiczeniami. Ponadto każdy z użytkowników będzie posiadał własne staystyki, możliwe do obserwacji na rysowanych wykresach. Usprawnieniem będzie możliwość obliczania swojego maksymalnego powtórzenie oraz progresji falowej. Niezabraknie również narzędzie do monitorowania zmiany wagii, które uzależnione jes od objętości treningowej oraz spożywanych kalorii.

\section{Istniejące rozwiązania.}

\section{Wymagania funkcjonalne}

\section{Wymagania niefunkcjonalne}
\chapter[Wykorzystane technologie.]{Wykorzystane technologie}

\section{Przegląd technologii}

\subsection{Swift (Objectiv-C)}
 
\subsection{Android (Java)}
 
\subsection{React Native (Javascript)} 

\section{Wybrane technologie}

\subsection{React Native + Redux}

\subsection{Javascript (JSX, ES6)}

\subsection{HTML  \& CSS}

\subsection{Firebase.}

\section{Wybrane technologie.}

\subsection{Microsoft Visual Studio.}

\subsection{React Native Debugger.}

\subsection{GIT- system kontorli wersji.}
\chapter[Realizacja aplikacji.]{Realizacja aplikacji}

\section{Architektura aplikacji mobilnej}

\section{Schemat działania aplikacji}

\section{Baza danych}

\section{Prezentacja aplikacji}

\section{Testy aplikacji}

\section{Wdrożenie}

\section{Obsługiwane systemy monitorowania}
\input{rozdzialy/Podsumowanie.tex}
%\input{rozdzialy/Bibliografia.tex}
%\input{rozdzialy/Podsumowanie.tex}

\nocite{*} %wszystkie wpisy w bibliografi
\bibliographystyle{unsrt} %{latex8} posortowane wzgledem wystepowania
\bibliography{bibliografia}%

%\addtocontents{toc}{\contentsline {chapter}{Bibliografia}{\thepage}{}}
\listoftables
%\addtocontents{toc}{\contentsline {chapter}{Spis tabel}{\thepage}{}}
\listoffigures
%\addtocontents{toc}{\contentsline {chapter}{Spis rysunków}{\thepage}{}}
\lstlistoflistings
%\addtocontents{toc}{\contentsline {chapter}{Spis listingów}{\thepage}{}}
%\listofalgorithms % w zaleznosci od kompilatora i wersji klasy moga wystapic bledy przy kompilacji
%\addtocontents{toc}{\contentsline {chapter}{Spis algorytmów}{\thepage}{}}

\newpage
\thispagestyle{empty}
\mbox{}
\biblioteka{tak} % tak/nie
\end{document}
