\chapter[Analiza wymagań.]{Analiza wymagń}

\section{Wizja oraz ogólny zarys aplikacji.}
Głównym celem powstania aplikacji jest pomoc osobom trenującym w organizacji ich treningu na siłowni. 

Często spotykanym zjawiskiem jest brak pamięci do ostatnio wykonywanego treningu. Zwłaszcza początkujący ale i również doświadczone osobymają problemy z zapamiętaniem trenowanych ćwiczeń, przenoszonego obciążenia czy też zakresu powtórzeń.

Trenujący próbują rozwiązać ten problem zabieraniem na trening zeszytu oraz notowaniem progresu.
NIe rozwiązuje to jednak do końca problemu. NIerzadko przytrafia się zapomnienie zeszytu, długopisu a same targanie wraz z innymi akcesoriami treningowymi nie należy do przyjemnych. Nie posiada on również nieograniczonego miejsca oraz jest podatny na uszkodzenia.

Najodpowiedniejszym rozwiązaniem problemów wydaje się aplikacja mobilna. Jedyną rzeczą jaką użytkownik musi zrobić, to wziąć swój telefon na trening. Telefon służy nam w różnych aspektach życia co sprawia, że trudno go zapomnieć.

Aplikacja nie będzie skomplikowana oraz przeładowana. Posiadać będzie czytelny interfejs. Najważniejszą przewidzianą funkcją ma być archiwizacja danych oraz notowanie aktualnego treningu. Ważnym dodatkiem będzie lista ćwczeń zawierająca opis każdego z nich. Zakładka ulubione ćwiczenia- pomoże łatwiej je pogrupować oraz odszukać użytkownikowi. Rzeczą usprawniającą trening będzie minutnik, który pomoże precyzyjnie odmieżać czas wykonywanego interwału lub przerwy między seriami lub ćwiczeniami. Ponadto każdy z użytkowników będzie posiadał własne staystyki, możliwe do obserwacji na rysowanych wykresach. Usprawnieniem będzie możliwość obliczania swojego maksymalnego powtórzenie oraz progresji falowej. Niezabraknie również narzędzie do monitorowania zmiany wagii, które uzależnione jes od objętości treningowej oraz spożywanych kalorii.

\section{Istniejące rozwiązania.}

\section{Wymagania funkcjonalne}

\section{Wymagania niefunkcjonalne}