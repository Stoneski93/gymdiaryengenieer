\chapter[Wstęp.]{Wstęp}

Pamięć bywa bardzo ulotna. Mózg człowieka jest w stanie myśleć o ograniczonej liczbie spraw. Naukowcy twierdzą, że ludzie mogą poświęcać swą uwagę maksymalnie czterem aspektom w ciągu jednej chwili. Przy dzisiejszym tempie życia niewielu z nas potrafi zapamiętać wielu ważnych informacji. Ciągły stres oraz gmach obowiązków zdecydowanie nie ułatwia tego zadania.

Wielką pomocą oraz ułatwieniem jest organizacja własnego dnia oraz notowanie różnych aspektów naszego życia. Od dawien ludzie używali podręcznych notesów czy też kalendarzy by planować ważne wydarzenia, spotkania czy nawet tak błache sprawy jak zakupy czy codzienne obowiązki.

Z biegiem lat dokonano wielu innowacji technologicznych. Współcześnie mnóstwo ludzi może cieszyć się rozmaitymi urządzeniami mobilnymi. Zasłużone notesy zostały wyparte przez smartfony czy tablety. Towarzyszą nam praktycznie w każdej sytuacji. Są bardzo wygodne, poręczne a co najważniejsze- dają wiele możliwości. Jedną z nich jest organizacja swoich przyszłych zadań lub planów.

Sportowcy również borykają się z tego rodzaju problemami. Zdarza się, że zapominają ostatnio wykonywanych przez siebie ćwiczeń treningowych, ilości obciążenia bądź seri ćwiczenia. Większość z nich do organizowania wysiłku używa zwykłych zeszytów, które często są zapominane czy też bardzo nieporęczne przy i tak wielu akcesoriach treningowych. Ponadto posiadają one ograniczone miejsce na dane. Na pomoc im wychodzi aplikacja mobilna pełniąca funkcję organizera. 

\section{Cel pracy}

Celem pracy jest stworzenie aplikacji mobilnej umożliwiającej zarządzanie treningiem siłowym.
Przeznaczenie programu na platformę smartfonów sprawi, że każdy użytkownik zabierając ze sobą swój telefon, weźmie również organizer. Mobilny dziennik treningowy udostępni wiele możliwości zarządzanie treningiem. Najważniejszą z nich będzie notowanie oraz archiwizacja wykonywanych ćwiczeń. Dodatkowo użytkownik będzie mógł przeglądać wybrane ćwiczenia oraz uczyć się poprawnej techniki ich wykonywania. Możliwe będzie dokumentowania zmian wagi oraz kalori, obliczania maksymalnego powtórzenia oraz progresji falowej. Aplikacja zachowa również wiele statystyk oraz rekordy wraz z datą ustanowienia. Dane będą zbierane w sieci, a organizer będzie wymagał autoryzacji użytkownika.

\section{Zakres pracy}

\begin {enumerate}
\item Analiza wymagań aplikacji
\item Przegląd istniejących rozwiązań
\item Przegląd dostępnych technologii
\item Zaprojektowanie aplikacji oraz wybór technlogii
\item Implementacja aplikacji oraz testy
\end {enumerate}

\textbf{Zakres prac projektowych oraz implementacyjnych aplikacji mobilnej}

\begin {itemize}
\item stworzenie szaty graficznej interfejsu użytkownika
\item implementacja logiki oraz zasady działania aplikacji
\item implementacja zdalnej bazy danych aplikacji
\item połączenie aplikacji z usługą operującą na bazie danych
\item implementacja zewnętrznej usługi do autoryzacji użytkowników
\item testowanie aplikacji mobilnej
\item wdrożenie gotowej aplikacji
\end {itemize}

\section{Plan pracy}

W \textbf{rozdziale 1} został zamieszczony cały wstęp do pracy wraz z opisanym celem pracy oraz zakresem prac. \textbf{Rozdział 2} przedstawia analizę wymagań. Ukazana jest tam wizja twórcy oraz ogólny zarys aplikacji. Ponadto zawiera on przykłady istniejących rozwiązań aplikacji podobnych do tworzonej. Wymagania funkcjonalne oraz nie funkcjonalne wraz z opisanym diagramem przypadków użycia. W \textbf{rozdziale 3} umieszczono przegląd technologii, w których możliwe byłoby wykonanie aplikacji. Następnym punktem są opisy technologi oraz narzędzi programistycznych użytch do implementacji danego pomysłu. \textbf{Rozdział 4} to realizacja aplikacji. Zawiera on architekture aplikacji, przedstawia jej schemat działania oraz bazę danych. Ponadto umieszono w nim prezentację wraz ze zdjęciami całej aplikacji, jej testy oraz wdrożenie. Dwa ostatnie czyli \textbf{rozdział 5} oraz \textbf{rozdział 6} to kolejno podsumowanie oraz bibliografia pracy.