\chapter*{Mobile application to manage training in the gym}

The main goal of this thesis was to create a mobile application that helps athletes manage their gym training sessions. The app allows for its users for an easy and convenient way to organize and track the progress of their workouts.

Human memory is an unreliable and athletes struggle with that as well. Trainees don’t remember the weight they have recently worked out with or how many series or repetitions they did.
 
Some write it down in a regular notebook. It is not very handy, though, might be forgotten at times and has limited space. Most people nowadays carry a mobile phone around everywhere, including gym. The application solves these and many others problems. It allows for a user to note progress during a training session that can be reviewed and used  at a later time.

Application has been developed using JavaScript framework React Native, supported by Redux technology. Framework uses functional approach to programming and frontends technologies that include HTML, JSX and CSS. Despite only web technologies being used, the application is 100\% native.

The paper contains five chapters. The first is an introduction. It presents the goal, plan and scope of work. Second chapter shows an overview of the application, existing solutions and functional and nonfunctional requirements. The most important is chapter four and it describes the development and implementation process. It contains application architecture, principle of operation, tests and product presentation.The last chapter is the summary.

Finally there’s an app used by advanced athletes.


