\chapter*{Mobile application to manage training in the gym}

Major goal of this thesis was to create a mobile application that helps athletes manage their trainings in the gym. For this purpose has been designed and developed mobile application. It is easier way to organize and track the progress of sportsmen workouts.
Humans memory is an unreliable. Athletes have a problem with that as well. Trainees don’t remember what weight recently trained or how series, reps they lifted. Somebody note that in a usual notebook. But It is not very handy, can be forgot it and have limited space.
That application solves these and many others problems. Most people have a mobile at this moment, which takes to the gym, too. Software allows to note progress during training and planning or observing some statistic.
Application is programmed in the JavaScript’s framework React Native supported by Redux technology. Framework uses functional approach to programming and frontends technologies like HTML, JSX and CSS. Although used web technologies, application is 100% native.
The paper contains five chapters. The first is an introduction. It presents the goal, plan and scope of work. Second chapter- shows an overview of the application, existing solutions and functional and nonfunctional requirements. The most important fourth chapter describes implementations project. It contains application architecture, principle of operation, tests and product presentations. The last is summary of the work.
Finally created the application, already used by athletes.

